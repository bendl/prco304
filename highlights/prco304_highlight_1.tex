\documentclass[11pt,a4paper]{article}

% font
\renewcommand{\familydefault}{\sfdefault}


\usepackage[margin=0.8in]{geometry}
\usepackage[utf8]{inputenc}
\usepackage{amsmath}
\usepackage{amsfonts}
\usepackage{amssymb}
\usepackage[hidelinks]{hyperref}
\usepackage{float}

\usepackage{lipsum}% http://ctan.org/pkg/lipsum

%% Bibliography/references packages
\usepackage[comma]{natbib}
%%\bibliographystyle{agsm}
\bibliographystyle{dcu}

%% https://en.wikibooks.org/wiki/LaTeX/List_Structures
\usepackage{scrextend}

% tables, row colour
\usepackage{tabularx,colortbl}

% https://tex.stackexchange.com/questions/22751/how-to-force-table-caption-on-top
%\usepackage[tableposition=top]{caption}
\usepackage{float}
\floatstyle{plaintop}
\restylefloat{table}

% https://en.wikibooks.org/wiki/LaTeX/List_Structures
\usepackage{enumitem}


%% Report variables
\newcommand{\scname}{BEN-1816}
\newcommand{\dlatestv}{1.00}

\usepackage{array,booktabs,arydshln,xcolor}
\usepackage{xcolor}% http://ctan.org/pkg/xcolor
\usepackage{fancyhdr}% http://ctan.org/pkg/fancyhdr
\fancypagestyle{main}{%
	\renewcommand{\headrulewidth}{2pt}
	\renewcommand{\headrule}{\hbox to\headwidth{%
		\color{red}\leaders\hrule height \headrulewidth\hfill}}
	\renewcommand{\footrulewidth}{2pt}
	\renewcommand{\footrule}{\hbox to\headwidth{%
		\color{red}\leaders\hrule height \headrulewidth\hfill}}
	
	%\fancyhf{}
	%\fancyhead[LE]{\textbf{\leftmark}}
	%\fancyhead[RE]{\textbf{\scname{}}}
	%\fancyhead[LO]{\textbf{\scname{}}}
	%\fancyhead[RO]{\textbf{\rightmark}}

	%\fancyfoot[LE]{\textbf{\thepage}}
	%\fancyfoot[RE]{\textbf{\scname{} Configuration Guide}}
	%\fancyfoot[LO]{\textbf{\scname{} Configuration Guide}}
	%\fancyfoot[RO]{\textbf{\thepage}}
}


%% Make bibliography show in table of contents
%% https://tex.stackexchange.com/questions/8458/making-the-bibliography-appear-in-the-table-of-contents
\usepackage[nottoc,numbib]{tocbibind}
%% ^^^ overwrites \bibname, so set it back
\renewcommand{\bibname}{References}

\RequirePackage{filecontents}
\begin{filecontents}{prco304.bib}
@online{wikipedia:dft,
  author = {Wikipedia},
  title = {Discrete Fourier transform},
  year = 2018,
  url = {https://en.wikipedia.org/wiki/Discrete\_Fourier\_transform},
  urldate = {2018-01-15}
}
@online{server:gpu,
  author = {Amazon AWS},
  title = {Introducing Amazon EC2 P2 Instances, the largest GPU-Powered virtual machine in the cloud},
  year = 2018,
  url = {https://aws.amazon.com/about-aws/whats-new/2016/09/introducing-amazon-ec2-p2-instances-the-largest-gpu-powered-virtual-machine-in-the-cloud/},
  urldate = {2016-09-26}
}
@misc{scarabhardware,
title={MiniSpartan6+}, 
journal={{Scarab Hardware}},
url={https://www.scarabhardware.com/minispartan6/},
year=2014
}
@misc{arty,
title={Arty Artix-7 FPGA Development Board}, 
journal={{Avnet}},
url={https://uk.rs-online.com/web/p/programmable-logic-development-kits/1346478/},
year=2015
}
@misc{arndt2002algorithms,
  title={Algorithms For Programmers},
  author={Arndt, J{\"o}rg},
  year = 2002
}
@book{hdl,
title={HDL Programming Fundamentals: VHDL and Verilog},
author={Nazeih Botros},
year={2006},
publisher={Da Vinci Engineering Press}
}

@misc{arm, title={ARM in the World of FPGA-Based Prototyping}, url={https://community.arm.com/processors/b/blog/posts/arm-in-the-world-of-fpga-based-prototyping}, journal={Arm Community},
year={2016}}

@book{microblaze,
title={MicroBlaze 
Processor Reference 
Guide},
journal={Xilinx},
year={2017}
} 

\end{filecontents}

%s comments
\usepackage{verbatim}

%inline graphs
\usepackage{wrapfig}
% multiple figures on line
\usepackage{subfig}

\usepackage{graphicx}
\graphicspath{ {img/} }

% Caption font size
% https://tex.stackexchange.com/questions/86120/font-size-of-figure-caption-header
\usepackage[font=scriptsize,labelfont=bf]{caption}

%\setlength{\belowcaptionskip}{-10pt}
%\setlength{\abovecaptionskip}{-5pt} % Chosen fairly arbitrarily


\usepackage{fancyhdr}
\pagestyle{fancy}
\lhead{\rightmark}
\chead{}
\rhead{FPGA-based Soft-Core CPU (Rev. \dlatestv{})}
\lfoot{Page \thepage}
\cfoot{}
\rfoot{Ben Lancaster 10424877}

\renewcommand{\subsectionmark}[1]{\markright{\thesubsection\ #1}}


\begin{document}

\begin{titlepage}
\begin{center}

\vspace*{5cm}
\Large
\textbf{
%%PRCO304 - Project Initiation Document
Highlight Reports
}

\vspace{0.4cm}
\large
%%Space optimised FPGA-based side-microprocessor.
PRCO304 - FPGA-based RISC microprocessor
%%EMBEDDED CPU - FPGA-based RISC microprocessor

\vspace{4cm}
\textbf{Ben Lancaster}\\
\today 


\end{center}

\end{titlepage}

\pagestyle{main}

\section*{Revision History}
\begin{table}[h]
\def\arraystretch{1.5}%  1 is the default, change whatever you need
    \begin{tabularx}{\textwidth}{|l|l|X|}
    \hline
    Date & Highlight & Changes \\
	\specialrule{2pt}{-2pt}{0pt}
	06/02/2018 & 1 & Highlight report 1.
	\\ \hline
    \end{tabularx}
    \caption{Document revisions.}
\end{table}
\newpage

\renewcommand*\contentsname{Table of Contents}
\tableofcontents
\newpage

\section{Highlight Reports}
\subsection{Highlight Report 1}
\begin{table}[H]
\def\arraystretch{1.5}%  1 is the default, change whatever you need
    \begin{tabularx}{\textwidth}{|X|}
    \hline 
	\multicolumn{1}{|c|}{\textbf{PRCO304: Highlight Report 1}}
    \\
	\specialrule{2pt}{-2pt}{0pt}
    \textbf{Name:} Ben Lancaster
    \\ \specialrule{2pt}{-2pt}{0pt}
	\textbf{Date:} 06/02/2018
	\\ \specialrule{2pt}{-2pt}{0pt}
	\textbf{Active project stage:} Stage 1.1:  Research  and  Requirement Gathering
	\\ \specialrule{2pt}{-2pt}{0pt}
	\textbf{Review of work undertaken:}\newline
	This week was assigned to work on stage 1.1:  Research  and  requirement
gathering. \newline\newline
	\textbf{Research and requirement gathering:}\newline
	Research into existing soft-core processor designs has been started to identify their features, targets, and advantages and disadvantages. Key existing soft-core processors found are:\newline
	-  Xilinx' MicroBlaze: a 32-bit Xilinx FPGA embeddable core capable of running operating systems, like Linux. Exposes a configurable GUI to customise the build of the processor to suit designers requirements (like number of GPIO, interrupts, timers, etc.).\newline
	- ARM Cortex-A9: a 32-bit Xilinx and Altera FPGA core. Features out-of-order execution, compatible with existing ARM Thumb2 C compilers, and multi-core processing.\newline\newline
	I have used this research to aim my soft-core processor's requirements and architecture. To document and finalize my processors design and requirements, I have started a processor specification and reference document. This document outlines the processors features, architecture, compatibility, and instructions.
	\newline\newline
	Additional progress:\newline
	- Version control set up for documentation, highlight reports, and code bases.
		
	\\ \hline
	\textbf{Risks and Challenges:}\newline
	{\color{red} Urgent risks:}\newline
	{\color{orange} New risks:}\newline
	{\color{purple} Existing risks:}\newline
	\\ \hline
	\textbf{Plan of work for the next week:}\newline
	Work will begin on Stage 1.2: Core high level design.\newline\newline
	Finalised specifications and architecture of the soft-core processor will be put into a processor specification and reference document.
	\newline
	Architecture, control, pipelines, will be visualised in this document.
	
	\\ \hline
	\textbf{Date(s) of supervisory meeting(s) since last Highlight:}\newline
	This is the 1st highlight report.\newline
	30/01/18 - An introductory meeting was held to discuss the project initiation document (PID) and gain feedback on the project.
	\\ \hline
	\textbf{Notes from supervisory meeting(s) held since last Highlight:}\newline
	Ensure risks are carefully explored and project core deliverables are realistic and achievable.
	\\ \hline
	\textbf{Stage review:}\newline
	N/A
	\\ \hline
    \end{tabularx}
    %\caption{Document revisions.}
\end{table}


\newpage
\bibliography{prco304} 
\end{document}