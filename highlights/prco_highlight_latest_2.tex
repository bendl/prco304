\documentclass[11pt,a4paper]{article}
%DIF LATEXDIFF DIFFERENCE FILE
%DIF DEL .\prco304_highlight_1.tex   Wed Feb  7 11:20:53 2018
%DIF ADD .\prco304_highlight_2.tex   Wed Feb 14 14:10:23 2018

% font
\renewcommand{\familydefault}{\sfdefault}


\usepackage[margin=0.8in]{geometry}
\usepackage[utf8]{inputenc}
\usepackage{amsmath}
\usepackage{amsfonts}
\usepackage{amssymb}
\usepackage[hidelinks]{hyperref}
\usepackage{float}

\usepackage{lipsum}% http://ctan.org/pkg/lipsum

%% Bibliography/references packages
\usepackage[comma]{natbib}
%%\bibliographystyle{agsm}
\bibliographystyle{dcu}

%% https://en.wikibooks.org/wiki/LaTeX/List_Structures
\usepackage{scrextend}

% tables, row colour
\usepackage{tabularx,colortbl}

% https://tex.stackexchange.com/questions/22751/how-to-force-table-caption-on-top
%\usepackage[tableposition=top]{caption}
\usepackage{float}
\floatstyle{plaintop}
\restylefloat{table}

% https://en.wikibooks.org/wiki/LaTeX/List_Structures
\usepackage{enumitem}

%DIF 37a37-38
% https://jansoehlke.com/2010/06/strikethrough-in-latex/ %DIF > 
\usepackage{ulem} %DIF > 
%DIF -------

%DIF 38a40
 %DIF > 
%DIF -------
%% Report variables
\newcommand{\scname}{BEN-1816}
\newcommand{\dlatestv}{1.00}

%DIF 42a45-48
\definecolor{babyblueeyes}{rgb}{0.63, 0.79, 0.95} %DIF > 
\definecolor{ballblue}{rgb}{0.13, 0.67, 0.8} %DIF > 
\definecolor{beaublue}{rgb}{0.74, 0.83, 0.9} %DIF > 
 %DIF > 
%DIF -------
\usepackage{array,booktabs,arydshln,xcolor}
\usepackage{xcolor}% http://ctan.org/pkg/xcolor
\usepackage{fancyhdr}% http://ctan.org/pkg/fancyhdr
\fancypagestyle{main}{%
	\renewcommand{\headrulewidth}{2pt}
	\renewcommand{\headrule}{\hbox to\headwidth{%
%DIF 48c55
%DIF < 		\color{red}\leaders\hrule height \headrulewidth\hfill}}
%DIF -------
		\color{babyblueeyes}\leaders\hrule height \headrulewidth\hfill}} %DIF > 
%DIF -------
	\renewcommand{\footrulewidth}{2pt}
	\renewcommand{\footrule}{\hbox to\headwidth{%
%DIF 51c58
%DIF < 		\color{red}\leaders\hrule height \headrulewidth\hfill}}
%DIF -------
		\color{babyblueeyes}\leaders\hrule height \headrulewidth\hfill}} %DIF > 
%DIF -------
	
	%\fancyhf{}
	%\fancyhead[LE]{\textbf{\leftmark}}
	%\fancyhead[RE]{\textbf{\scname{}}}
	%\fancyhead[LO]{\textbf{\scname{}}}
	%\fancyhead[RO]{\textbf{\rightmark}}

	%\fancyfoot[LE]{\textbf{\thepage}}
	%\fancyfoot[RE]{\textbf{\scname{} Configuration Guide}}
	%\fancyfoot[LO]{\textbf{\scname{} Configuration Guide}}
	%\fancyfoot[RO]{\textbf{\thepage}}
}


%% Make bibliography show in table of contents
%% https://tex.stackexchange.com/questions/8458/making-the-bibliography-appear-in-the-table-of-contents
\usepackage[nottoc,numbib]{tocbibind}
%% ^^^ overwrites \bibname, so set it back
\renewcommand{\bibname}{References}

\RequirePackage{filecontents}

%s comments
\usepackage{verbatim}

%inline graphs
\usepackage{wrapfig}
% multiple figures on line
\usepackage{subfig}

\usepackage{graphicx}
\graphicspath{ {img/} }

% Caption font size
% https://tex.stackexchange.com/questions/86120/font-size-of-figure-caption-header
\usepackage[font=scriptsize,labelfont=bf]{caption}

%\setlength{\belowcaptionskip}{-10pt}
%\setlength{\abovecaptionskip}{-5pt} % Chosen fairly arbitrarily


\usepackage{fancyhdr}
\pagestyle{fancy}
\lhead{\rightmark}
\chead{}
\rhead{FPGA-based Soft-Core CPU (Rev. \dlatestv{})}
\lfoot{Page \thepage}
\cfoot{}
\rfoot{Ben Lancaster 10424877}

\renewcommand{\subsectionmark}[1]{\markright{\thesubsection\ #1}}
%DIF PREAMBLE EXTENSION ADDED BY LATEXDIFF
%DIF CCHANGEBAR PREAMBLE %DIF PREAMBLE
\RequirePackage[dvips]{changebar} %DIF PREAMBLE
\RequirePackage{color}\definecolor{RED}{rgb}{1,0,0}\definecolor{BLUE}{rgb}{0,0,1} %DIF PREAMBLE
\providecommand{\DIFaddtex}[1]{\protect\cbstart{\protect\color{blue}#1}\protect\cbend} %DIF PREAMBLE
\providecommand{\DIFdeltex}[1]{\protect\cbdelete{\protect\color{red}#1}\protect\cbdelete} %DIF PREAMBLE
%DIF SAFE PREAMBLE %DIF PREAMBLE
\providecommand{\DIFaddbegin}{} %DIF PREAMBLE
\providecommand{\DIFaddend}{} %DIF PREAMBLE
\providecommand{\DIFdelbegin}{} %DIF PREAMBLE
\providecommand{\DIFdelend}{} %DIF PREAMBLE
%DIF FLOATSAFE PREAMBLE %DIF PREAMBLE
\providecommand{\DIFaddFL}[1]{\DIFadd{#1}} %DIF PREAMBLE
\providecommand{\DIFdelFL}[1]{\DIFdel{#1}} %DIF PREAMBLE
\providecommand{\DIFaddbeginFL}{} %DIF PREAMBLE
\providecommand{\DIFaddendFL}{} %DIF PREAMBLE
\providecommand{\DIFdelbeginFL}{} %DIF PREAMBLE
\providecommand{\DIFdelendFL}{} %DIF PREAMBLE
%DIF HYPERREF PREAMBLE %DIF PREAMBLE
\providecommand{\DIFadd}[1]{\texorpdfstring{\DIFaddtex{#1}}{#1}} %DIF PREAMBLE
\providecommand{\DIFdel}[1]{\texorpdfstring{\DIFdeltex{#1}}{}} %DIF PREAMBLE
\newcommand{\DIFscaledelfig}{0.5}
%DIF HIGHLIGHTGRAPHICS PREAMBLE %DIF PREAMBLE
\RequirePackage{settobox} %DIF PREAMBLE
\RequirePackage{letltxmacro} %DIF PREAMBLE
\newsavebox{\DIFdelgraphicsbox} %DIF PREAMBLE
\newlength{\DIFdelgraphicswidth} %DIF PREAMBLE
\newlength{\DIFdelgraphicsheight} %DIF PREAMBLE
% store original definition of \includegraphics %DIF PREAMBLE
\LetLtxMacro{\DIFOincludegraphics}{\includegraphics} %DIF PREAMBLE
\newcommand{\DIFaddincludegraphics}[2][]{{\color{blue}\fbox{\DIFOincludegraphics[#1]{#2}}}} %DIF PREAMBLE
\newcommand{\DIFdelincludegraphics}[2][]{% %DIF PREAMBLE
\sbox{\DIFdelgraphicsbox}{\DIFOincludegraphics[#1]{#2}}% %DIF PREAMBLE
\settoboxwidth{\DIFdelgraphicswidth}{\DIFdelgraphicsbox} %DIF PREAMBLE
\settoboxtotalheight{\DIFdelgraphicsheight}{\DIFdelgraphicsbox} %DIF PREAMBLE
\scalebox{\DIFscaledelfig}{% %DIF PREAMBLE
\parbox[b]{\DIFdelgraphicswidth}{\usebox{\DIFdelgraphicsbox}\\[-\baselineskip] \rule{\DIFdelgraphicswidth}{0em}}\llap{\resizebox{\DIFdelgraphicswidth}{\DIFdelgraphicsheight}{% %DIF PREAMBLE
\setlength{\unitlength}{\DIFdelgraphicswidth}% %DIF PREAMBLE
\begin{picture}(1,1)% %DIF PREAMBLE
\thicklines\linethickness{2pt} %DIF PREAMBLE
{\color[rgb]{1,0,0}\put(0,0){\framebox(1,1){}}}% %DIF PREAMBLE
{\color[rgb]{1,0,0}\put(0,0){\line( 1,1){1}}}% %DIF PREAMBLE
{\color[rgb]{1,0,0}\put(0,1){\line(1,-1){1}}}% %DIF PREAMBLE
\end{picture}% %DIF PREAMBLE
}\hspace*{3pt}}} %DIF PREAMBLE
} %DIF PREAMBLE
\LetLtxMacro{\DIFOaddbegin}{\DIFaddbegin} %DIF PREAMBLE
\LetLtxMacro{\DIFOaddend}{\DIFaddend} %DIF PREAMBLE
\LetLtxMacro{\DIFOdelbegin}{\DIFdelbegin} %DIF PREAMBLE
\LetLtxMacro{\DIFOdelend}{\DIFdelend} %DIF PREAMBLE
\DeclareRobustCommand{\DIFaddbegin}{\DIFOaddbegin \let\includegraphics\DIFaddincludegraphics} %DIF PREAMBLE
\DeclareRobustCommand{\DIFaddend}{\DIFOaddend \let\includegraphics\DIFOincludegraphics} %DIF PREAMBLE
\DeclareRobustCommand{\DIFdelbegin}{\DIFOdelbegin \let\includegraphics\DIFdelincludegraphics} %DIF PREAMBLE
\DeclareRobustCommand{\DIFdelend}{\DIFOaddend \let\includegraphics\DIFOincludegraphics} %DIF PREAMBLE
\LetLtxMacro{\DIFOaddbeginFL}{\DIFaddbeginFL} %DIF PREAMBLE
\LetLtxMacro{\DIFOaddendFL}{\DIFaddendFL} %DIF PREAMBLE
\LetLtxMacro{\DIFOdelbeginFL}{\DIFdelbeginFL} %DIF PREAMBLE
\LetLtxMacro{\DIFOdelendFL}{\DIFdelendFL} %DIF PREAMBLE
\DeclareRobustCommand{\DIFaddbeginFL}{\DIFOaddbeginFL \let\includegraphics\DIFaddincludegraphics} %DIF PREAMBLE
\DeclareRobustCommand{\DIFaddendFL}{\DIFOaddendFL \let\includegraphics\DIFOincludegraphics} %DIF PREAMBLE
\DeclareRobustCommand{\DIFdelbeginFL}{\DIFOdelbeginFL \let\includegraphics\DIFdelincludegraphics} %DIF PREAMBLE
\DeclareRobustCommand{\DIFdelendFL}{\DIFOaddendFL \let\includegraphics\DIFOincludegraphics} %DIF PREAMBLE
%DIF END PREAMBLE EXTENSION ADDED BY LATEXDIFF

\begin{document}

\begin{titlepage}
\begin{center}

\vspace*{5cm}
\Large
\textbf{
%%PRCO304 - Project Initiation Document
Highlight Reports
}

\vspace{0.4cm}
\large
%%Space optimised FPGA-based side-microprocessor.
PRCO304 - FPGA-based RISC microprocessor
%%EMBEDDED CPU - FPGA-based RISC microprocessor

\vspace{4cm}
\textbf{Ben Lancaster}\\
\today 


\end{center}

\end{titlepage}

\pagestyle{main}

\section*{Revision History}
\begin{table}[h]
\def\arraystretch{1.5}%  1 is the default, change whatever you need
    \begin{tabularx}{\textwidth}{|l|l|X|}
    \hline
    Date & Highlight & Changes \\
	\specialrule{2pt}{-2pt}{0pt}
	\DIFaddbeginFL \DIFaddFL{15/02/2018 }& \DIFaddFL{2 }& \DIFaddFL{Highlight report 2. }\\ \hline
	\DIFaddendFL 06/02/2018 & 1 & Highlight report 1. \\ \hline
    \end{tabularx}
    \caption{Document revisions.}
\end{table}
\newpage

\renewcommand*\contentsname{Table of Contents}
\tableofcontents
\newpage

\section{Highlight Reports}
\subsection{Highlight Report 1}
\begin{table}[H]
\def\arraystretch{1.5}%  1 is the default, change whatever you need
    \begin{tabularx}{\textwidth}{|X|}
    \hline 
	\multicolumn{1}{|c|}{\textbf{PRCO304: Highlight Report 1}}
    \\
	\specialrule{2pt}{-2pt}{0pt}
    \textbf{Name:} Ben Lancaster
    \\ \specialrule{2pt}{-2pt}{0pt}
	\textbf{Date:} 06/02/2018
	\\ \specialrule{2pt}{-2pt}{0pt}
	\textbf{Active project stage:} Stage 1.1:  Research  and  Requirement Gathering
	\\ \specialrule{2pt}{-2pt}{0pt}
	\textbf{Review of work undertaken:}\newline
	This week was assigned to work on stage 1.1:  Research  and  requirement
gathering. \newline\newline
	\textbf{Research and requirement gathering:}\newline
	Research into existing soft-core processor designs has been started to identify their features, targets, and advantages and disadvantages. Key existing soft-core processors found are:\newline
	-  Xilinx' MicroBlaze: a 32-bit Xilinx FPGA embeddable core capable of running operating systems, like Linux. Exposes a configurable GUI to customise the build of the processor to suit designers requirements (like number of GPIO, interrupts, timers, etc.).\newline
	- ARM Cortex-A9: a 32-bit Xilinx and Altera FPGA core. Features out-of-order execution, compatible with existing ARM Thumb2 C compilers, and multi-core processing.\newline\newline
	I have used this research to aim my soft-core processor's requirements and architecture. To document and finalize my processors design and requirements, I have started a processor specification and reference document. This document outlines the processors features, architecture, compatibility, and instructions.
	\newline\newline
	Additional progress:\newline
	- Version control set up for documentation, highlight reports, and code bases.

	\\ \hline
	\textbf{Risks and Challenges:}\newline
	{\color{red} Urgent risks:}\newline
	{\color{orange} New risks:}\newline
	{\color{purple} Existing risks:\newline
	RC4: Schedule overrun. A gantt time chart has been created to better visualize task durations and requirements.}\newline
	\\ \hline
	\textbf{Plan of work for the next week:}\newline
	Work will begin on Stage 1.2: Core high level design.\newline\newline
	Finalised specifications and architecture of the soft-core processor will be put into a processor specification and reference document.
	\newline
	Architecture, control, pipelines, will be visualised in this document.

	\\ \hline
	\textbf{Date(s) of supervisory meeting(s) since last Highlight:}\newline
	This is the 1st highlight report.\newline
	30/01/18 - An introductory meeting was held to discuss the project initiation document (PID) and gain feedback on the project.
	\\ \hline
	\textbf{Notes from supervisory meeting(s) held since last Highlight:}\newline
	Ensure risks are carefully explored and project core deliverables are realistic and achievable.
	\\ \hline
    \end{tabularx}
    %\caption{Document revisions.}
\end{table}


\DIFaddbegin \subsection{\DIFadd{Highlight Report 2}}
\begin{table}[H]
\def\arraystretch{1.5}%DIF >   1 is the default, change whatever you need
    \begin{tabularx}{\textwidth}{|X|}
    \hline 
	\multicolumn{1}{|c|}{\textbf{PRCO304: Highlight Report 2}}
    \\
	\specialrule{2pt}{-2pt}{0pt}
    \textbf{\DIFaddFL{Name:}} \DIFaddFL{Ben Lancaster
    }\\ \specialrule{2pt}{-2pt}{0pt}
	\textbf{\DIFaddFL{Date:}} \DIFaddFL{15/02/2018
	}\\ \specialrule{2pt}{-2pt}{0pt}
	\textbf{\DIFaddFL{Active project stage:}} \DIFaddFL{Stage 1.2:  Core high level design
	}\\ \specialrule{2pt}{-2pt}{0pt}
	\textbf{\DIFaddFL{Review of work undertaken:}}\newline
	\DIFaddFL{This week was assigned to work on stage 1.2:  Core high level design.
gathering. }\newline\newline
	\textbf{\DIFaddFL{Core high level design:}}\newline
	\DIFaddFL{I have spent this week defining a processor specification and creating a processor specification/reference guide booklet (see attached). This booklet will contain both high-level and technical details regarding the design and implementation of the processor, including: register sets, control and pipelining strategies, the ISA and each instruction, and the compiler and how to use it.	
	}\newline\newline
	\DIFaddFL{This booklet will be developed over the life cycle of the project. Although the specification has been clearly defined, the booklet will be incrementally updated as processor features/requirements are added to the implementation (such as instructions, modules, and compiler features).
	}\newline\newline
	\DIFaddFL{Currently the reference booklet contains: register set definitions, several primitive instructions, and a brief introduction to instruction cycle timing.
	}

	\\ \hline
	\textbf{\DIFaddFL{Risks and Challenges:}}\newline
	{\color{red} \DIFaddFL{Urgent risks:}}\newline
	{\color{orange} \DIFaddFL{New risks:}}\newline
	{\color{purple} \DIFaddFL{Existing risks:}\newline
	\sout{RC4: Schedule overrun. A gantt time chart has been created to better visualize task durations and requirements.}}\newline
	{\color{gray} \DIFaddFL{Resolved risks:}\newline
	\DIFaddFL{RC4: Schedule overrun. A gantt time chart has been created to better visualize task durations and requirements. (See attached time management chart.)}}\newline
	\\ \hline
	\textbf{\DIFaddFL{Plan of work for the next week:}}\newline
	\DIFaddFL{Work will begin on Stage 2.0: Core dev. Register set implementation.}\newline\newline
	\DIFaddFL{The register set module will be implemented in Verilog for the processor. Unit tests will be created to verify the timing/behaviour of the module.
	}\newline
	\DIFaddFL{The processor specification/reference booklet will be updated to describe how the register set has been implemented in the processor.
	}

	\\ \hline
	\textbf{\DIFaddFL{Date(s) of supervisory meeting(s) since last Highlight:}}\newline
	\DIFaddFL{This is the 1st highlight report.}\newline
	\DIFaddFL{30/01/18 - An introductory meeting was held to discuss the project initiation document (PID) and gain feedback on the project.
	}\\ \hline
	\textbf{\DIFaddFL{Notes from supervisory meeting(s) held since last Highlight:}}\newline
	\DIFaddFL{Ensure risks are carefully explored and project core deliverables are realistic and achievable.
	}\\ \hline
    \end{tabularx}
    %DIF > \caption{Document revisions.}
\end{table}

 \DIFaddend\end{document}
