
\subsubsection*{NOP}
\label{isa_nop}
\begin{description}[align=right,labelwidth=4cm]
\item [Description] The NOP instruction performs no action for 1 instruction cycle.
\item [Assembly] NOP
\item [Pseudocode]
\item [Registers altered] None
\item [Clock cycles] 2 (FETCH, DECODE)
\end{description}

\begin{table}[h]
\def\arraystretch{1.3}%  1 is the default, change whatever you need
    \begin{tabularx}{\textwidth}{|p{4cm}|X|}
    \hline
    15:11 & 10:0 \\
	\specialrule{2pt}{-2pt}{0pt}
	00000 & X
	\\ \hline
    \end{tabularx}
\end{table}


\subsubsection*{LW - Load Word}\label{isa_lw}
\begin{description}[align=right,labelwidth=4cm]
\item [Description] Copies a 16-bit word from RAM to a register.
\item [Assembly] LW Rd, +4(Ra)
\item [Pseudocode] Rd $<=$ RAM[Ra + Simm5]
\item [Registers altered] Rd
\item [Clock cycles] 6 (FETCH, DECODE, READ, EXECUTE, RAM, WRITE)
\end{description}

\begin{table}[H]
\def\arraystretch{1.3}%  1 is the default, change whatever you need
    \begin{tabularx}{\textwidth}{|p{4cm}|p{2cm}|p{2cm}|X|}
    \hline
    15:11 & 10:8 & 7:5 & 4:0 \\
	\specialrule{2pt}{-2pt}{0pt}
	00001 & Rd & Ra & Simm5
	\\ \hline
    \end{tabularx}
\end{table}


\subsubsection*{SW - Store Word}\label{isa_sw}
\begin{description}[align=right,labelwidth=4cm]
\item [Description] Copies a 16-bit from a register to RAM.
\item [Assembly] SW Rd, +4(Ra)
\item [Pseudocode] RAM[Ra+Simm5] $<=$ Rd
\item [Registers altered] None
\item [Clock cycles] 6 (FETCH, DECODE, READ, EXECUTE, RAM, WRITE)
\end{description}

\begin{table}[H]
\def\arraystretch{1.3}%  1 is the default, change whatever you need
    \begin{tabularx}{\textwidth}{|p{4cm}|p{2cm}|p{2cm}|X|}
    \hline
    15:11 & 10:8 & 7:5 & 4:0 \\
	\specialrule{2pt}{-2pt}{0pt}
	00001 & Rd & Ra & Simm5
	\\ \hline
    \end{tabularx}
\end{table}

\subsubsection*{MOVR}
\begin{description}[align=right,labelwidth=4cm]
\item [Description] Copies a 16-bit register value to another register.
\item [Assembly] MOVR \%Ra, \%Rd 
\item [Pseudocode] Rd $<=$ Ra
\item [Registers altered] Rd
\item [Clock cycles] 5 (FETCH, DECODE, READ, EXECUTE, WRITE)
\end{description}

\begin{table}[H]
\def\arraystretch{1.3}%  1 is the default, change whatever you need
    \begin{tabularx}{\textwidth}{|p{4cm}|p{2cm}|p{2cm}|X|}
    \hline
    15:11 & 10:8 & 7:5 & 4:0 \\
	\specialrule{2pt}{-2pt}{0pt}
	00011 & Rd & Ra & X
	\\ \hline
    \end{tabularx}
\end{table}

\subsubsection*{MOVI}\label{isa_movi}
\begin{description}[align=right,labelwidth=4cm]
\item [Description] Copies an 8-bit immediate to a Register
\item [Assembly] MOVR \%Ra, \%Rd 
\item [Pseudocode] Rd $<=$ Imm8
\item [Registers altered] Rd
\item [Clock cycles] 5 (FETCH, DECODE, READ, EXECUTE, WRITE)
\end{description}

\begin{table}[H]
\def\arraystretch{1.3}%  1 is the default, change whatever you need
    \begin{tabularx}{\textwidth}{|p{4cm}|p{3cm}|X|}
    \hline
    15:11 & 10:8 & 7:0 \\
	\specialrule{2pt}{-2pt}{0pt}
	00100 & Rd & Imm8
	\\ \hline
    \end{tabularx}
\end{table}


\subsubsection*{ADD}
\begin{description}[align=right,labelwidth=4cm]
\item [Description] Add the value of register Ra to Rd.
\item [Assembly] ADD \%Rd, \%Ra
\item [Pseudocode]Rd $<=$ Rd + Ra
\item [Registers altered] Rd
\item [Clock cycles] 5 (FETCH, DECODE, READ, EXEC, WRITE)
\end{description}

\begin{table}[H]
\def\arraystretch{1.3}%  1 is the default, change whatever you need
    \begin{tabularx}{\textwidth}{|p{4cm}|p{2cm}|p{2cm}|X|}
    \hline
    15:11 & 10:8 & 7:5 & 4:0 \\
	\specialrule{2pt}{-2pt}{0pt}
	01000 & Rd & Ra & X
	\\ \hline
    \end{tabularx}
\end{table}

\subsubsection*{ADDI}
\begin{description}[align=right,labelwidth=4cm]
\item [Description] Adds an immediate value to a destination register, Rd.
\item [Assembly] ADDI \$255, \%Rd
\item [Pseudocode]Rd $<=$ Rd + Imm8
\item [Registers altered] Rd
\item [Clock cycles] 5 (FETCH, DECODE, READ, EXEC, WRITE)
\end{description}

\begin{table}[H]
\def\arraystretch{1.3}%  1 is the default, change whatever you need
    \begin{tabularx}{\textwidth}{|p{4cm}|p{3cm}|X|}
    \hline
    15:11 & 10:8 & 7:0 \\
	\specialrule{2pt}{-2pt}{0pt}
	01001 & Rd & Imm8
	\\ \hline
    \end{tabularx}
\end{table}


\subsubsection*{SUBI}
\begin{description}[align=right,labelwidth=4cm]
\item [Description] Subtracts an immediate value from a destination register, Rd.
\item [Assembly] SUBI \$255, \%Rd
\item [Pseudocode]Rd $<=$ Rd - Imm8
\item [Registers altered] Rd
\item [Clock cycles] 5 (FETCH, DECODE, READ, EXEC, WRITE)
\end{description}

\begin{table}[H]
\def\arraystretch{1.3}%  1 is the default, change whatever you need
    \begin{tabularx}{\textwidth}{|p{4cm}|p{3cm}|X|}
    \hline
    15:11 & 10:8 & 7:0 \\
	\specialrule{2pt}{-2pt}{0pt}
	01001 & Rd & Imm8
	\\ \hline
    \end{tabularx}
\end{table}


\subsubsection*{CMP}
\label{isa:cmp}
\begin{description}[align=right,labelwidth=4cm]
\item [Description] Sets status register bits depending on the result of Rd - Ra
\item [Assembly] CMP Rd, Ra
\item [Pseudocode]{\nameref{sect:core_regs_sr}} $<=$ CMP(Ra, Rb)
\item [Registers altered] Rd
\item [Clock cycles] 5 (FETCH, DECODE, READ, EXEC, WRITE)
\end{description}

\begin{table}[H]
\def\arraystretch{1.3}%  1 is the default, change whatever you need
    \begin{tabularx}{\textwidth}{|p{4cm}|p{2cm}|p{2cm}|X|}
    \hline
    15:11 & 10:8 & 7:5 & 4:0 \\
	\specialrule{2pt}{-2pt}{0pt}
	01101 & Rd & Ra &  X
	\\ \hline
    \end{tabularx}
\end{table}


\subsubsection*{SETC}
\begin{description}[align=right,labelwidth=4cm]
\item [Description] Set register Rd to 0 or 1 depending on the Status Register and Immediate value.
\item [Assembly] SETC \$0x08, \%Rd
\item [Pseudocode]Rd $<$= 1 if Imm8 and Status Register equal, else 0.
\item [Registers altered] Rd
\item [Clock cycles] 5 (FETCH, DECODE, READ, EXEC, WRITE)
\end{description}

\begin{table}[H]
\def\arraystretch{1.3}%  1 is the default, change whatever you need
    \begin{tabularx}{\textwidth}{|p{4cm}|p{3cm}|X|}
    \hline
    15:11 & 10:8 & 7:0 \\
	\specialrule{2pt}{-2pt}{0pt}
	10101 & Rd & Imm8 (See JMP Imm8)
	\\ \hline
    \end{tabularx}
\end{table}

\newpage
\subsubsection*{JMP}\label{isa_jmp}
\begin{description}[align=right,labelwidth=4cm]
\item [Description] Jumps the Program Counter (PC) if the condition is met within the Status Register.
\item [Assembly] JMP Rd, Imm8
\item [Pseudocode] PC $<=$ Rd if Status Register \& Imm8).
\item [Registers altered] None
\item [Clock cycles] 5 (FETCH, DECODE, READ, EXEC, BRANCH)
\end{description}

\begin{table}[H]
\def\arraystretch{1.3}%  1 is the default, change whatever you need
    \begin{tabularx}{\textwidth}{|p{4cm}|p{3cm}|X|}
    \hline
    15:11 & 10:8 & 7:0 \\
	\specialrule{2pt}{-2pt}{0pt}
	01100 & Rd & Imm8
	\\ \hline
    \end{tabularx}
\end{table}
An 8 bit immediate (7-0) can be set in the JMP instruction to create conditional jumps.

\begin{table}[h]
	\def\arraystretch{1.3}%  1 is the default, change whatever you need
    \begin{tabularx}{\textwidth}{|c|c|c|c|c|X|X|}
    \hline
    & 15-11 & 10-8 & \multicolumn{2}{c|}{7-0} & Semantics & Status Register \\
    \specialrule{2pt}{-2pt}{0pt}
    JMP		& 01100 & Rd & \multicolumn{2}{c|}{0000 0000} & Unconditional Jump & Any\\ \hline
    JE		& 01100 & Rd & \multicolumn{2}{c|}{0000 0001} & Jump Equal & ZF=1\\ \hline
    JNE		& 01100 & Rd & \multicolumn{2}{c|}{0000 0010} & Jump Not Equal & ZF=0\\ \hline
    JG	& 01100 & Rd & \multicolumn{2}{c|}{0000 0011} & Jump Greater Than & ZF=0 and SF=OF\\ \hline
    JGE		& 01100 & Rd & \multicolumn{2}{c|}{0000 0100} & Jump Greater Than or Equal & SF=OF\\ \hline
    JL		& 01100 & Rd & \multicolumn{2}{c|}{0000 0101} & Jump Less Than & SF$<>$OF\\ \hline
    JLE		& 01100 & Rd & \multicolumn{2}{c|}{0000 0110} & Jump Less Than or Equal & ZF=1 or SF$<>$OF\\ \hline
    JS		& 01100 & Rd & \multicolumn{2}{c|}{0000 0111} & Jump Signed & SF=1\\ \hline
    JNS		& 01100 & Rd & \multicolumn{2}{c|}{0000 1000} & Jump Not Signed & SF=0 \\ \hline
    \end{tabularx}
    \caption{Conditional jump immediate bits}
\end{table}